Laplace Transforms (Eng. Maths 3)

\begin{itemize}
\item $\mathcal{L}[x] = \frac{1}{s^2}$
\item $\mathcal{L}[x^2] = \frac{2}{s^3}$
\item $\mathcal{L}[x^3] = \frac{6}{s^4}$
\item $\mathcal{L}[sin(ax)] = \frac{a^2}{s^2 + a^2}$
\item $\mathcal{L}[cos(ax)] = \frac{s^2}{s^2 + a^2}$
\end{itemize}


\section{Question 1}
Choose appropriate Laplace Transform from tables
%----------------------------------------%
\subsection{2011 Yellow}
$\mathcal{L}[cos3t + sin3t]$ = $\mathcal{L}[cos3t]$+ $\mathcal{L}[sin3t]$\\
Answer : \[ G(s) = \frac{s}{s^2+9}+\frac{3}{s^2+9}= \frac{s+3}{s^2+9}\]
%----------------------------------------%
\subsection{2011 Green}
$\mathcal{L}[cos32 + sin2t]$ = $\mathcal{L}[cos2t]$+ $\mathcal{L}[sin2t]$ \\

Answer : \[ G(s) = \frac{s}{s^2+4}+\frac{2}{s^2+4}= \frac{s+2}{s^2+4}\]
%----------------------------------------%

\subsection{2010 Yellow}

$\mathcal{L}[cosh2t + sinh2t]$ = $\mathcal{L}[cosh2t]$- $\mathcal{L}[sinh2t]$\\
Answer : \[ G(s) = \frac{s}{s^2-4}-\frac{2}{s^2-4}= \frac{s-2}{s^2-4}\]\\
This doesnt correspond to any of the answers! (Simplify it)\\
Answer : \[ G(s) = \frac{s-2}{s^2-4} = \frac{s-2}{(s-2)(s+2)} = \frac{1}{s+2}

%----------------------------------------%
\subsection{2010 Green}

%----------------------------------------%
%-----------------------------------------------------------------------------%
\section{Question 2}
%----------------------------------------%
\subsection{Q2 2011 Yellow}
$\mathcal{L}[e^{4t}(5t-1)]$ = $5\mathcal{L}[e^{4t}t]$ - $\mathcal{L}[e^{4t}]$
%----------------------------------------%
\subsection{Q2 2011 Green}
$\mathcal{L}[e^{-5t}(2t+1)]$ = $2\mathcal{L}[e^{-5t}t]$ + $\mathcal{L}[e^{-5t}]$
%----------------------------------------%
\subsection{Q2 2010 Yellow}

$\mathcal{L}[e^{3t}(t+1)]$ = $5\mathcal{L}[e^{3t}t]$ + $\mathcal{L}[e^{3t}]$
%----------------------------------------%
\subsection{Q2 2010 Green}

$\mathcal{L}[e^{-2t}(t+2)]$ = $5\mathcal{L}[e^{-2t}t]$ + 2$\mathcal{L}[e^{-2t}]$

%----------------------------------------%
%-----------------------------------------------------------------------------%
\section{Question 3}
%----------------------------------------%
\subsection{2011 Yellow}
$g(t) = sin(2t - 6)u_3(t)$ is in form $f(t-a)u_a(t)$ , with $a=3$ \\
If $f(t-a) = f(t-3) = sin(2t - 6)$ then $f(t)= sin(2t)$ \\
$\mathcal{L}[f(t-a)u_a(t)]$ =$e^{-as}F(s)$ \\
$F(s)$ is simply $\mathcal{L}[f(t)]$\\
%----------------------------------------%
\subsection{2011 Green}
$g(t) = cos(t-2)u_2(t)$ is in form $f(t-a)u_a(t)$ , with $a=2$ \\
If $f(t-a) = f(t-2) = cos(t-2)$ then $f(t)= cos(t)$ \\
$\mathcal{L}[f(t-a)u_a(t)]$ =$e^{-as}F(s)$ \\
$F(s)$ is simply $\mathcal{L}[f(t)]$\\
%----------------------------------------%
\subsection{2010 Yellow}
$g(t) = cos(t-3)u_3(t)$ is in form $f(t-a)u_a(t)$ , with $a=3$ \\
If $f(t-a) = f(t-3) = cos(t-3)$ then $f(t)= cos(t)$ \\
$\mathcal{L}[f(t-a)u_a(t)]$ =$e^{-as}F(s)$ \\
$F(s)$ is simply $\mathcal{L}[f(t)]$\\
%----------------------------------------%
\subsection{2010 Green}
$g(t) = sin(2t - 2)u_1(t)$ is in form $f(t-a)u_a(t)$ , with $a=1$ \\
If $f(t-a) = f(t-1) = sin(2t - 2)$ then $f(t)= sin(2t)$ \\
$\mathcal{L}[f(t-a)u_a(t)]$ =$e^{-as}F(s)$ \\
$F(s)$ is simply $\mathcal{L}[f(t)]$\\
%----------------------------------------%

%-----------------------------------------------------------------------------%
\section{Question 4}
%----------------------------------------%
\subsection{Q4 2011 Yellow}

\begin{itemize}
\item Find the inverse Laplace transform of 
\[ G(s) = e^{-2s}\frac{1}{s^2} \]
\item Can consider G(s) in the following form , with $a=2$
\[ G(s) = e^{-as}F(s) \]
\item The inverse of this is
\[ g(t) = f(t-a) u_a(t) \]
\item Table Entry 2
\[ F(s) = 1/s^2 \arrow f(t) = t \] 
\item Identify $f(t-a)$
\[ \therefore f(t-a) = f(t-2) = t-2 \]
\item Answer
\[ g(t) = f(t-a)u_a(t) = (t-2)u_2(t) \]
\end{itemize}
%----------------------------------------%
\subsection{Q4 2011 Green}
\begin{itemize}
\item Find the inverse Laplace transform of 
\[ G(s) = \frac{s}{s^2 + 4s + 4} \]
\item Factorise the denominator?  $(s+2)^2$
\item
\[ G(s) = \frac{s}{(s+2)^2} \]
\item We can express the numerator as follows:  $s = (s+2)-2$
\item
\[ G(s) = \frac{s}{(s+2)^2} =  \frac{s+2}{(s+2)^2} - \frac{2}{(s+2)^2} \]
\item The first term simplifies to something simple. 
\item The second term can be solved using Table Entry 5, with $a=-1$ and $n=1$.
\item
\[ g(t)  = e{-2t} - e{-2t}t \]
\end{itemize}

%----------------------------------------%
\subsection{Q4 2010 Yellow}

\begin{itemize}
\item Find the inverse Laplace transform of 
\[ G(s) = \frac{s}{s^2 -2s + 1} \]
\item Factorise the denominator?  $(s-1)^2$
\item
\[ G(s) = \frac{s}{(s-1)^2} \]
\item We can express the numerator as follows:  $s = (s-1)+1$
\item
\[ G(s) = \frac{s}{(s-1)^2} =  \frac{s-1}{(s-1)^2} + \frac{1}{(s-1)^2} \]
\item The first term simplifies to something simple. 
\item The second term can be solved using Table Entry 5, with $a=1$ and $n=1$.
\item Answer:
\[ g(t)  = e{t} + e{t}t = e{t}(1+t)  \]

\end{itemize}
%----------------------------------------%
\subsection{Q4 2010 Green}
\begin{itemize}
\item Find the inverse Laplace transform of 
\[ G(s) = \frac{s}{s^2 -2s + 1} \]
\item Factorise the denominator?  $(s-1)^2$
\item
\[ G(s) = \frac{s}{(s-1)^2} \]
\item We can express the numerator as follows:  $s = (s-1)+1$
\item
\[ G(s) = \frac{s}{(s-1)^2} =  \frac{s-1}{(s-1)^2} + \frac{1}{(s-1)^2} \]
\item The first term simplifies to something simple. 
\item The second term can be solved using Table Entry 5, with $a=1$ and $n=1$.
\item Answer:
\[ g(t)  = e{t} + e{t}t = e{t}(1+t)  \]

\end{itemize}


%-------------------------------------------------------------------------%
\section{Question 5}
%----------------------------------------%
\subsection{Q5 2011 Yellow}
\begin{itemize}
\item Find the inverse Laplace transform of 
\[ G(s) = \frac{s-1}{s^2 - 2s + 2} \]
\item Factorise the denominator?  not this time. ( no obvious roots)
\item
\[ G(s) = \frac{s-1}{s^2 + 2s + 2} = \frac{s-1}{(s-1)^2 + 1} \]
\item We can express the transform in following form:  $F(s-a)$ where $a=1$ 
\item Necessarily \[ F(s) = = \frac{s}{s^2 + 1} \]
\item The Inverse Laplace transform of this: $f(t) = cos(t)$
\item The overal Inverse Laplace transform of this: \[g(t) = e^{at}f(t) = e^tcos(t)\]
\end{itemize}

%----------------------------------------%
\subsection{Q5 2011 Green}
\begin{itemize}
\item Find the inverse Laplace transform of 
\[ G(s) = \frac{s-2}{s^2 + s -6} \]
\item Factorise the denominator?  $(s-2)(s+3)$
\item Simplification of the expression is possible
\[ G(s) = \frac{s-2}{(s-2)(s+3)} = \frac{1}{s+3} \]
\item The inverse Laplace Transform can be computed using Table Entry 4
\[ g(t)  = e{-3t}  \]

\end{itemize}
%----------------------------------------%
\subsection{Q5 2010 Yellow}
\begin{itemize}
\item Find the inverse Laplace transform of 
\[ G(s) = \frac{s-2}{s^2 + s + 6} \]
\item Factorise the denominator?  $(s+2)(s+3)$
\item Can arrange expression in a way that Table Entries 8 and 9 can be used intuitively ($a=-3, b=-2$)
\[ G(s) = \frac{s-2}{(s+2)(s+3)} = \left(\frac{s}{(s+2)(s+3)} \right)-2\left(\frac{1}{(s+2)(s+3)}\right) \]
\{itemize}
%----------------------------------------%
\subsection{Q5 2010 Green}
\begin{itemize}
\item Find the inverse Laplace transform of 
\[ G(s) = \frac{s-2}{s^2 - s - 6} \]
\item Factorise the denominator?  $(s+2)(s-3)$
\item Can arrange expression in a way that Table Entries 8 and 9 can be used intuitively ($a=3, b=-2$)
\[ G(s) = \frac{s-2}{(s+2)(s-3)} = \left(\frac{s}{(s+2)(s-3)} \right)-2\left(\frac{1}{(s+2)(s-3)}\right) \]
\{itemize}
%----------------------------------------%
\section{Question 6}

\subsection{Q6 2011 Yellow}
\begin{itemize}
\item $f_1(t)=t^2$ : Laplace Transform $F_1(s)={2 \over s^3}$
\item $f_2(t)=t^2$ : Laplace Transform $F_2(s)={2 \over s^3}$
\item $\mathcal{L}[g(t)] = F_1(s)\times F_2(s) = {4 \over s^6}$
\item $g(t)$ = $\mathcal{L}^{-1}[{4 \over s^6}]$
\item Using Table Entry 3: $n$ is necessarily $5$
\end{itemize}

%----------------------------------------%
\subsection{Q6 2011 Green}
\begin{itemize}
\item $f_1(t)=e^{-t}$ : Laplace Transform $F_1(s)={1 \over s+1}$
\item $f_2(t)=e^{-t}$ : Laplace Transform $F_2(s)={1 \over s+1}$
\item $\mathcal{L}[g(t)] = F_1(s)\times F_2(s) = {1 \over (s+1)^2}$
\item $g(t)$ = $\mathcal{L}^{-1}[{1 \over (s+1)^2}]$
\item Using Table Entry 4: $n$ is necessarily $1$ , and $a=-1$
\item Answer $g(t) = te^{-t}$
\end{itemize}

%----------------------------------------%
\subsection{Q6 2010 Yellow}

\begin{itemize}
\item $f_1(t)=t$ : Laplace Transform $F_1(s)={1 \over s^2}$
\item $f_2(t)=t^2$ : Laplace Transform $F_2(s)={2 \over s^3}$
\item $\mathcal{L}[g(t)] = F_1(s)\times F_2(s) = {2 \over s^5}$
\item $g(t)$ = $\mathcal{L}^{-1}[{2 \over s^5}]$
\item Using Table Entry 3: $n$ is necessarily $4$
\item Remember to use a scaling coefficient $k$ in conjunction with the formula.
\item $G(s) = k { 2 \over s^{4+1}}$
\end{itemize}
\subsection{2010 Green}
%----------------------------------------%
\newpage
%-------------------------------------------------------------------------%
\section{Question 7}

\begin{itemize}
\item The period of a function $p$ may be identified by the following term : $f(t) = f(t+p)$
\item Period of a function is inversely proportional to the coefficient.
\item If $f(x)$ has period $p$ , then $f(2x)$ has period $p/2$
\item For older questions a periodic trigonometric function of form $trig(kx)$ has period $2\pi/k$
\end{itemize}
%----------------------------------------%
\subsection{Q7 2011 Yellow}
\begin{itemize}
\item If $f(x)$ has period $2$ , then $f(2x)$ has period $1$
\end{itemize}
%----------------------------------------%
\subsection{Q7 2011 Green}
\begin{itemize}
\item If $f(x)$ has period $2\pi$ , then $f(2x)$ has period $\pi$
\end{itemize}
%----------------------------------------%
\subsection{Q7 2010 Yellow}
\begin{itemize}
\item  $trig(kx)$ $k$= ${pi\over2}$
\item  $p$ = $\frac{2\pi}{pi\over2}$ =4
\end{itemize}
%----------------------------------------%
\subsection{Q7 2010 Green}
\begin{itemize}
\item  $trig(kx)$ $k$= ${1\over2}$
\item  $p$ = $\frac{2\pi}{1\over2}$ = $4\pi$
\end{itemize}
%----------------------------------------%
\newpage
\subsection{Q8 2010 Green}
\begin{itemize}
\item $f(x) = x - x^5$ 
\item $f(-1) = (-1) - (-1)^5$ = 0 
\item $f(1) = (1) - (1)^5$ = 0 
\item use a different number instead
\item $f(-1/2) = (-1/2) - (-1/2)^5$ = 0 
\item $f(1/2) = (1/2) - (1/2)^5$ = 0 
\item $f(x)$ is odd

\item $g(x) = x^2 sin x$
\item $g(-1) = (-1^2) \times sin(-1)$ 
\item $g(1) = (1^2) \times sin(-1)$ 
\item $g(x)$ is also odd
\end{itemize}
%----------------------------------------%
