%\documentclass[11pt, a4paper,dalthesis]{report}    % final
%\documentclass[11pt,a4paper,dalthesis]{report}
%\documentclass[11pt,a4paper,dalthesis]{book}

\documentclass[11pt,a4paper,titlepage,oneside,openany]{article}

\pagestyle{plain}
%\renewcommand{\baselinestretch}{1.7}

\usepackage{setspace}
%\singlespacing
\onehalfspacing
%\doublespacing
%\setstretch{1.1}

\usepackage{amsmath}
\usepackage{amssymb}
\usepackage{amsthm}
\usepackage{multicol}

\usepackage[margin=3cm]{geometry}
\usepackage{graphicx,psfrag}%\usepackage{hyperref}
\usepackage[small]{caption}
\usepackage{subfig}

\usepackage{algorithm}
\usepackage{algorithmic}
\newcommand{\theHalgorithm}{\arabic{algorithm}}

\usepackage{varioref} %NB: FIGURE LABELS MUST ALWAYS COME DIRECTLY AFTER CAPTION!!!
%\newcommand{\vref}{\ref}

\usepackage{index}
\makeindex
\newindex{sym}{adx}{and}{Symbol Index}
%\newcommand{\symindex}{\index[sym]}
%\newcommand{\symindex}[1]{\index[sym]{#1}\hfill}
\newcommand{\symindex}[1]{\index[sym]{#1}}

%\usepackage[breaklinks,dvips]{hyperref}%Always put after varioref, or you'll get nested section headings
%Make sure this is after index package too!
%\hypersetup{colorlinks=false,breaklinks=true}
%\hypersetup{colorlinks=false,breaklinks=true,pdfborder={0 0 0.15}}


%\usepackage{breakurl}

\graphicspath{{./images/}}

\usepackage[subfigure]{tocloft}%For table of contents
\setlength{\cftfignumwidth}{3em}

\input{longdiv}
\usepackage{wrapfig}


%\usepackage{index}
%\makeindex
%\usepackage{makeidx}

%\usepackage{lscape}
\usepackage{pdflscape}
\usepackage{multicol}

\usepackage[utf8]{inputenc}

%\usepackage{fullpage}

%Compulsory packages for the PhD in UL:
%\usepackage{UL Thesis}
\usepackage{natbib}

%\numberwithin{equation}{section}
\numberwithin{equation}{section}
\numberwithin{algorithm}{section}
\numberwithin{figure}{section}
\numberwithin{table}{section}
%\newcommand{\vec}[1]{\ensuremath{\math{#1}}}

%\linespread{1.6} %for double line spacing

\usepackage{afterpage}%fingers crossed

\newtheorem{thm}{Theorem}[section]
\newtheorem{defin}{Definition}[section]
\newtheorem{cor}[thm]{Corollary}
\newtheorem{lem}[thm]{Lemma}

%\newcommand{\dbar}{{\mkern+3mu\mathchar'26\mkern-12mu d}}
\newcommand{\dbar}{{\mkern+3mu\mathchar'26\mkern-12mud}}

\newcommand{\bbSigma}{{\mkern+8mu\mathsf{\Sigma}\mkern-9mu{\Sigma}}}
\newcommand{\thrfor}{{\Rightarrow}}

\newcommand{\mb}{\mathbb}
\newcommand{\bx}{\vec{x}}
\newcommand{\bxi}{\boldsymbol{\xi}}
\newcommand{\bdeta}{\boldsymbol{\eta}}
\newcommand{\bldeta}{\boldsymbol{\eta}}
\newcommand{\bgamma}{\boldsymbol{\gamma}}
\newcommand{\bTheta}{\boldsymbol{\Theta}}
\newcommand{\balpha}{\boldsymbol{\alpha}}
\newcommand{\bmu}{\boldsymbol{\mu}}
\newcommand{\bnu}{\boldsymbol{\nu}}
\newcommand{\bsigma}{\boldsymbol{\sigma}}
\newcommand{\bdiff}{\boldsymbol{\partial}}

\newcommand{\tomega}{\widetilde{\omega}}
\newcommand{\tbdeta}{\widetilde{\bdeta}}
\newcommand{\tbxi}{\widetilde{\bxi}}



\newcommand{\wv}{\vec{w}}

\newcommand{\ie}{i.e. }
\newcommand{\eg}{e.g. }
\newcommand{\etc}{etc}

\newcommand{\viceversa}{vice versa}
\newcommand{\FT}{\mathcal{F}}
\newcommand{\IFT}{\mathcal{F}^{-1}}
%\renewcommand{\vec}[1]{\boldsymbol{#1}}
\renewcommand{\vec}[1]{\mathbf{#1}}
\newcommand{\anged}[1]{\langle #1 \rangle}
\newcommand{\grv}[1]{\grave{#1}}
\newcommand{\asinh}{\sinh^{-1}}

\newcommand{\sgn}{\text{sgn}}
\newcommand{\morm}[1]{|\det #1 |}

\newcommand{\galpha}{\grv{\alpha}}
\newcommand{\gbeta}{\grv{\beta}}
%\newcommand{\rnlessO}{\mb{R}^n \setminus \vec{0}}
\usepackage{listings}

\interfootnotelinepenalty=10000

\newcommand{\sectionline}{%
  \nointerlineskip \vspace{\baselineskip}%
  \hspace{\fill}\rule{0.5\linewidth}{.7pt}\hspace{\fill}%
  \par\nointerlineskip \vspace{\baselineskip}
}

\renewcommand{\labelenumii}{\roman{enumii})}

\begin{document}

%----------------------------------------%
\section*{Question 1 part A i}
{\large

\[ (u_2(t) -  u_{\infty}(t)) \times [t-2]\]
$u_{\infty}(t)$=0

\[g(t) \mbox{ in the form }u_a(t)  \times f(t-a) \mbox{i.e. in the form } u_2(t)  \times f(t-2)\]

\begin{itemize}
\item $f(t-2) = t-2$ therefore $f(t) = t$.
\item Also $a=2$
\item We will use Table Entry 17.
\end{itemize}


\[ F(s)= \frac{1}{s^2} \]


\[ G(s)  = e^{-as}F(s)= \frac{e^{-2s}}{s^2} \]
}
%----------------------------------------%
\section*{Question 1 part A ii }
{ \large
\[ g(t) = \left[u_o(t) -  u_3(t)\right] sin(\pi t)=[1 - u_3(t)] sin (\pi t)\]

\begin{itemize}
\item We need to re-express $sin (\pi t)$ in form $f(t-3)$
\item Remark: t = (t-3)+3
\item $sin (\pi t) = sin (\pi(t-3)+ 3\pi) $
\item We can use the trigonometric idenity sin(A+B) =sinA cos B + cosA sin B
\item We will specifically look at sin B and cos B, i.e. sin $(3\pi)$ and cos $(3\pi)$. 
\item sin $(3\pi)$ = 0 and cos $(3\pi)$ = -1.
\item Therefore \item $sin (\pi t) = sin (\pi(t-3)+ 3\pi)  = -sin (\pi(t-3) $
\end{itemize}
\[ g(t) = sin (\pi t) + u_3(t)sin(\pi(t-3) \]
\begin{itemize}
\item The first term is straightforward
\item The second term is the first term shifted by 3 time units.
\end{itemize}
}
%----------------------------------------%
\section*{Question 1 part A iii }
{\large
The base function B(s)

\[\int^{p}_{0} f(t)e^{-st}dt\]

Can be re-expressed as: 
\[\int^{1}_{0} 5e^{-st}dt + \int^{2}_{1} 4e^{-st}dt \]

\[5\int^{1}_{0} e^{-st}dt + 4\int^{2}_{1} e^{-st}dt \]

\[\int e^{-st}dt = \frac{e^{-st}}{-s}\]


This will give us the Laplace transform of the Base function.%----------------------------------------%
Another way of solving this:

\[ (u_0(t) -  u_1(t))  \times 5 + (u_1(t) -  u_2(t))  \times 4 = 5u_0(t) -  5u_1(t) + 4u_1(t) -  4u_2(t)\]

\[= 5 u_o(t) -  u_1(t) - 4u_2(t) = 5 -  u_1(t) - 4u_2(t)  \]

The Laplace transform of the Base function is
\[B(s) = \frac{5}{s}- \frac{e^{-s}}{s} - \frac{4e^{-2s}}{s}  \]

N.B. This is Quadratic in nature 
\[B(s) = \frac{(4e^{-s}-5)(e^{-s}+1)}{s}  \]

The Laplace transform of the given function is
\[G(s) = \frac{B(s)}{1-e^{-sp}} = \frac{B(s)}{1-e^{-2s}}  \]

The denominator is quadratic in nature too. ( difference of squares)
\[G(s) = \frac{(4e^{-s}-5)(e^{-s}+1)}{(s)(1-e^{-s})(1+e^{-s})} =  \frac{4e^{-s}-5}{(s)(1-e^{-s})} \]

Partial Fraction Expansion may help simplify further
}
\newpage
%----------------------------------------%
\section*{Question 1 part b}

\textbf{Inverse Laplace Transforms : } \\
Find the inverse Laplace transforms of the following 2,3,3

\subsection*{Question 1 part b i}
\[\frac{2s-2}{s^2-s-6} = \frac{2s-2}{(s-3)(s+2)} = \frac{2s}{(s-3)(s+2)}-\frac{2}{(s-3)(s+2)}\]
Using Table Entries 8 and 9, with a=3 and b=-2. solve on the board.

\subsection*{Question 1 part b ii}
\[\frac{e^{-s}}{s^2 + 4s + 4}= \frac{e^{-s}}{(s+2)^2}  \]

\begin{itemize}
\item Laplace transform is in form $e{-as} \times F(S)$, with $a =1$
\item See Table Entry 17 (between Heaviside and Ramp Functions)
\item First we find the inverse Laplace Transform of F(S).
\[F(S) = \frac{1}{(s+2)^2}  \]
\item $F(s)$ is in form
\[F(S) = k \frac{n!}{(s-a)^{n+1}}  \]
Necessarily $n=1$, $a=-2$ and $k=1$ k is not necessary and we will drop it.
\item The inverse laplace transform of F(S) is therefore
\[ \mathcal{L}^{-1}[F(s)] = t^n e^{at}\]
\[ f(t) \mathcal{L}^{-1}[\frac{1}{(s+2)^2}] = t e^{-2t}\]
\item Now to find $g(t)$
\[g(t) = u_a(t)\times f(t-a)= u_1(t)\times f(t-1) = u_1(t)\times (t-1) e^{-2t-2}\]
\end{itemize}

\subsection*{Question 1 part b iii}
\[ G(s) = tan^1(s+2)\]
\begin{itemize}
\item Q1 part b iii usually employs table entry 18
%\[ \mathcal{L}^{-1}[t f(T)] = -F^{\prime}(s)\]
%\item That is to say ; G(s) is the derivative of something (times -1)
%\item Recall that
%\[ \frac{d}{dx}\left( \frac{1}{1+s^2}\right)  = tan^{-1}(x)\]
%\item
\end{itemize}

\newpage

\section*{Convolution}
\textbf{Notation:}
\[
\mathcal{L}^{-1} [F(s)] = f(t) \]
We can compute $ (f * g )(t)$, the convolution of two functions $f(t)$ and $g(t)$, by following these steps:\\
\bigskip
%\normalsize
\begin{itemize}
\item Get the Laplace transforms of the two component functions : $\mathcal{L}[f(t)] = F(s)$ and $\mathcal{L}[g(t)] = G(s)$
\item Multiply these two Laplace transforms: $ F(s) \times G(s)$
\item Find the inverse Laplace transform of the product: $\mathcal{L}^{-1}[F(s) \times G(s)] $
%\item The inverse Laplace transform is the convolution, which is the result we are looking for.
\end{itemize}
\newpage
\section*{Question 2}
%----------------------------------------%
\begin{itemize}
\item Differential Equations
\item Integral Equations (using Convolution)
\end{itemize}
%----------------------------------------%
\section*{Question 2 Part A}
\begin{itemize}
\item Table Entry 12: $ \mathcal{L}[y^{\prime}(t)]=$ $sY(s)- y(0)$
\item Table Entry 13: $ \mathcal{L}[y^{\prime\prime}(t)]=$ $s^2Y(s)- sy(0) - y^{\prime}(0)$
\item Be Mindful of Boundary Conditions. In most past papers, they are zero, but not this year.
\item $y(0) = 1$ $y^{\prime}(0) = 0$

\[ y^{\prime\prime} +4y = 5e^{-t} \]

\item Finding the Laplace Transform of everything
\[(s^2Y(s)- s) - (4Y(s)) = \frac{5}{s+1} \]

\item Simplifying the RHS
\[(s^2-4)Y(s) = \frac{5}{s+1} + s  = \frac{5}{s+1} + \frac{s^2+s}{s+1} = \frac{s^2+s +5}{s+1} \]

\[(s^2-4)Y(s) = \frac{s^2+s +5}{s+1} \]

\[Y(s) = \frac{s^2+s +5}{(s+1)(s-2)(s+2)} \]
\end{itemize}
Partial Fraction Expansion to solve
%----------------------------------------%
\section*{Question 2 Part B}
\[ y(t) = sin 3t - 2\int^t_0 cos 3(t-u)y(u)du\]
\[ y(t) = sin 3t - 2\left[cos(3t)\ast y(t)\right]\]
Find the Laplace transform for both sides
\[ Y(s) = \frac{3}{s^2+9} - 2\left[\frac{s}{s^2+9} \times  Y(s)\right]\]

\[ Y(s) = \frac{3-2sY(s)}{s^2+9} \]
Cross Multiplication

\[ Y(s)(s^2+9) = 3-2sY(s) \]

\[ Y(s)(s^2+2s+9) = 3 \]

\[ Y(s) = \frac{3}{s^2+2s+9} = \frac{3}{(s+1)^2+8} \]

Let us look at this matric is form (see Table Entry 17)

$F(s-a) \mbox{i.e. F(s+1)} \frac{3}{(s+1)^2+8}$

Necessarily \[ F(s) = \frac{3}{s^2+8} \]
%----------------------------------------%


\newpage
\newpage
$f(t) = t-2$



\[u_2(t) -  u_{\infty}(t) \times [t-2]\]


%----------------------------------------%

\[u_o(t) -  u_3(t) sin(\pi t)=1 - u_3(t) sin (\pi t)\]

%----------------------------------------%

\[u_0(t) -  u_1(t)  \times 5 + u_1(t) -  u_2(t)  \times 4 = 5u_0(t) -  5u_1(t) + 4u_1(t) -  4u_2(t)\]

\[= 5u_o(t) -  u_1(t) - 4u_2(t)\]

%----------------------------------------%
\section*{Question 1 part ii}

\textbf{Inverse Laplace Transforms : } \\
Find the inverse Laplace transforms of the following 2,3,3

$\frac{2s-2}{s^2-s+6}$
$\frac{e^{-s}}{s^2 + 4s + 4}$
$ tan^1(s+2)$




\textbf{Notation:}
\[
\mathcal{L}^{-1} [F(s)] = f(t) \]
We can compute $ (f * g )(t)$, the convolution of two functions $f(t)$ and $g(t)$, by following these steps:\\
\bigskip
%\normalsize
\begin{itemize}
\item Get the Laplace transforms of the two component functions : $\mathcal{L}[f(t)] = F(s)$ and $\mathcal{L}[g(t)] = G(s)$
\item Multiply these two Laplace transforms: $ F(s) \times G(s)$
\item Find the inverse Laplace transform of the product: $\mathcal{L}^{-1}[F(s) \times G(s)] $
%\item The inverse Laplace transform is the convolution, which is the result we are looking for.
\end{itemize}

\section*{Question 2}
%----------------------------------------%
\begin{itemize}
\item Differential Equations
\item Integral Equations (using Convolution)
\end{itemize}
%----------------------------------------%
\begin{itemize}
\item Table Entry 12: $ y^{\prime\prime}(t)$
\item Table Entry 13: $ y^{\prime\prime}(t)$
\end{itemize}
%----------------------------------------%

\[ y(t) = sin 3t - 2\int^t_0 cos 3(t-u)y(u)du\]
\[ y(t) = sin 3t - 2\left[cos 3\ast y(t)\right]\]
Find the Laplace transform for both sides
\[ Y(s) = \frac{3}{s^2+9} - 2\left[\frac{s}{s^2+9} \times  Y(s)\right]\]
%----------------------------------------%
\end{document} 