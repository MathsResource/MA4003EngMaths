% Inverse Laplace Transforms
\begin{document}
%---------------------------------------------%
\frame{
\textbf{Inverse Laplace Transforms : } \\

\textbf{Notation:}
\[ 
\mathcal{L}^{-1} [F(s)] = f(t) \]
}
%---------------------------------------------%
\frame{


}
%---------------------------------------------%

Compute the inverse laplace transform of  \frac{s}{s^2-2s+1}
\mathcal{L}^{-1} [ {s \over (s-a)(s-b)}] =
We can't use this formula.
Factorize the denominator:
\frac{s}{s^2-2s+1} = \frac{s}{(s-1)^2}
\frac{s}{(s-1)^2} = \frac{(s-1) +1 }{(s-1)^2} = \frac{s-1 }{(s-1)^2} + \frac{1 }{(s-1)^2}

Try a different approach

f(t)  = (1+t)e^t
%------------------------------------------------------------------------%
\frac{s-2}{s^2+5s+6} = \frac{s-2}{(s+2)(s+3)}
Split this into two terms
\frac{s-2}{(s+5)(s+6)} = \frac{s}{(s+2)(s+3)} - \frac{2}{(s+2)(s+3)}
Get the inverse Laplace transfrom of each term
\frac{s}{(s+5)(s+6)} :
a = -2
b = -3
a-b = (-2)-(-3) = 1
\frac{a \over a-b} ={-2 \over 1}
\frac{b \over a-b} ={-3 \over 1}
