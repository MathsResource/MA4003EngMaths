\documentclass[a4]{beamer}
\usepackage{amssymb}
\usepackage{graphicx}
\usepackage{subfigure}
\usepackage{newlfont}
\usepackage{amsmath,amsthm,amsfonts}
%\usepackage{beamerthemesplit}
\usepackage{pgf,pgfarrows,pgfnodes,pgfautomata,pgfheaps,pgfshade}
\usepackage{mathptmx}  % Font Family
\usepackage{helvet}   % Font Family
\usepackage{color}

\mode<presentation> {
 \usetheme{Default} % was Frankfurt
 \useinnertheme{rounded}
 \useoutertheme{infolines}
 \usefonttheme{serif}
 %\usecolortheme{wolverine}
% \usecolortheme{rose}
\usefonttheme{structurebold}
}

\setbeamercovered{dynamic}

\title[MathsCast]{MathsCasts - Dynamic Maths Support) \\ {\normalsize Statistics}}
\author[Kevin O'Brien]{Kevin O'Brien \\ {\scriptsize Kevin.obrien@ul.ie}}
\date{Summer 2011}
\institute[Maths \& Stats]{Dept. of Mathematics \& Statistics, \\ University \textit{of} Limerick}

\renewcommand{\arraystretch}{1.5}

\begin{document}

\begin{frame}
\titlepage
\end{frame}

\section{Descriptive Statistics using a Spreadsheet}
\frame{

Statistics divides the study of data into three parts:
\vspace{0.2cm}
\begin{itemize}
  \item Producing data;
  \item Organising and describing data (Descriptive statistics);
  \item Drawing conclusions from data (Statistical Inference);
\end{itemize}

\vspace{0.2cm}

The type of data determines which methods of analysis are appropriate and valid.
The major distinction is between \textit{quantitative} (numeric) and \textit{qualitative }(categorical) data.

}

\frame{\frametitle{Types of Data}

\begin{description}
  \item[Category data - Nominal] Identified by names or categories and cannot be
  organised according to any natural order \textit{e.g.} gender, colour of eyes, birthplace.
  \item[Category data - Ordinal] Identified by categories which can be ordered in
  some way \textit{e.g.} social class (lower, middle, upper), smoker status (light, moderate,
  heavy), stress levels (low, medium, high).
  \item[Numeric data] Indicates how much or how many \textit{e.g.} age, height, number of visits to
  the doctor annually.
  \item[Numeric data - Continuous numeric data] Data which can assume an infinite number of values (including decimal places)
  between any two given values \textit{e.g.} height.
  \item[Numeric data - Discrete numeric data] Data that can only have a finite number of numeric values (whole numbers only) \textit{e.g.}
  family size.
\end{description}

}

\frame{
\begin{center}
\includegraphics[width=\textheight]{Question1}
\end{center}
}

\frame{
\begin{center}
\includegraphics[width=\textheight]{Question2}
\end{center}
}

\frame{
\begin{center}
\includegraphics[width=\textheight]{Question3}
\end{center}
}

%--------------------------------------------------------------------------------%


\begin{frame}\frametitle{Outline of the Survey}
The objective of the survey is to obtain an assessment of the views or opinions of students studying in the Faculty of Business and Accounting studies at a specific university.

\vspace{0.4cm}

The Survey is broken into three parts - A,B and C. \\ \vspace{0.2cm}

A - Questions in this section are of ``Likert'' type. The data obtained here is ordinal (Categorical) although we treat it as if it were interval (Numerical) for the analysis.\\
\vspace{0.2cm}
B - One question asking people to indicate what School they are from - nominal (Categorical) data.\\
\vspace{0.2cm}
C - Another Likert question.
\end{frame}


\end{document}
