%\documentclass[11pt, a4paper,dalthesis]{report}    % final
%\documentclass[11pt,a4paper,dalthesis]{report}
%\documentclass[11pt,a4paper,dalthesis]{book}

\documentclass[11pt,a4paper,titlepage,oneside,openany]{article}

\pagestyle{plain}
%\renewcommand{\baselinestretch}{1.7}

\usepackage{setspace}
%\singlespacing
\onehalfspacing
%\doublespacing
%\setstretch{1.1}

\usepackage{amsmath}
\usepackage{amssymb}
\usepackage{amsthm}
\usepackage{multicol}

\usepackage[margin=3cm]{geometry}
\usepackage{graphicx,psfrag}%\usepackage{hyperref}
\usepackage[small]{caption}
\usepackage{subfig}

\usepackage{algorithm}
\usepackage{algorithmic}
\newcommand{\theHalgorithm}{\arabic{algorithm}}

\usepackage{varioref} %NB: FIGURE LABELS MUST ALWAYS COME DIRECTLY AFTER CAPTION!!!
%\newcommand{\vref}{\ref}

\usepackage{index}
\makeindex
\newindex{sym}{adx}{and}{Symbol Index}
%\newcommand{\symindex}{\index[sym]}
%\newcommand{\symindex}[1]{\index[sym]{#1}\hfill}
\newcommand{\symindex}[1]{\index[sym]{#1}}

%\usepackage[breaklinks,dvips]{hyperref}%Always put after varioref, or you'll get nested section headings
%Make sure this is after index package too!
%\hypersetup{colorlinks=false,breaklinks=true}
%\hypersetup{colorlinks=false,breaklinks=true,pdfborder={0 0 0.15}}


%\usepackage{breakurl}

\graphicspath{{./images/}}

\usepackage[subfigure]{tocloft}%For table of contents
\setlength{\cftfignumwidth}{3em}

\input{longdiv}
\usepackage{wrapfig}


%\usepackage{index}
%\makeindex
%\usepackage{makeidx}

%\usepackage{lscape}
\usepackage{pdflscape}
\usepackage{multicol}

\usepackage[utf8]{inputenc}

%\usepackage{fullpage}

%Compulsory packages for the PhD in UL:
%\usepackage{UL Thesis}
\usepackage{natbib}

%\numberwithin{equation}{section}
\numberwithin{equation}{section}
\numberwithin{algorithm}{section}
\numberwithin{figure}{section}
\numberwithin{table}{section}
%\newcommand{\vec}[1]{\ensuremath{\math{#1}}}

%\linespread{1.6} %for double line spacing

\usepackage{afterpage}%fingers crossed

\newtheorem{thm}{Theorem}[section]
\newtheorem{defin}{Definition}[section]
\newtheorem{cor}[thm]{Corollary}
\newtheorem{lem}[thm]{Lemma}

%\newcommand{\dbar}{{\mkern+3mu\mathchar'26\mkern-12mu d}}
\newcommand{\dbar}{{\mkern+3mu\mathchar'26\mkern-12mud}}

\newcommand{\bbSigma}{{\mkern+8mu\mathsf{\Sigma}\mkern-9mu{\Sigma}}}
\newcommand{\thrfor}{{\Rightarrow}}

\newcommand{\mb}{\mathbb}
\newcommand{\bx}{\vec{x}}
\newcommand{\bxi}{\boldsymbol{\xi}}
\newcommand{\bdeta}{\boldsymbol{\eta}}
\newcommand{\bldeta}{\boldsymbol{\eta}}
\newcommand{\bgamma}{\boldsymbol{\gamma}}
\newcommand{\bTheta}{\boldsymbol{\Theta}}
\newcommand{\balpha}{\boldsymbol{\alpha}}
\newcommand{\bmu}{\boldsymbol{\mu}}
\newcommand{\bnu}{\boldsymbol{\nu}}
\newcommand{\bsigma}{\boldsymbol{\sigma}}
\newcommand{\bdiff}{\boldsymbol{\partial}}

\newcommand{\tomega}{\widetilde{\omega}}
\newcommand{\tbdeta}{\widetilde{\bdeta}}
\newcommand{\tbxi}{\widetilde{\bxi}}



\newcommand{\wv}{\vec{w}}

\newcommand{\ie}{i.e. }
\newcommand{\eg}{e.g. }
\newcommand{\etc}{etc}

\newcommand{\viceversa}{vice versa}
\newcommand{\FT}{\mathcal{F}}
\newcommand{\IFT}{\mathcal{F}^{-1}}
%\renewcommand{\vec}[1]{\boldsymbol{#1}}
\renewcommand{\vec}[1]{\mathbf{#1}}
\newcommand{\anged}[1]{\langle #1 \rangle}
\newcommand{\grv}[1]{\grave{#1}}
\newcommand{\asinh}{\sinh^{-1}}

\newcommand{\sgn}{\text{sgn}}
\newcommand{\morm}[1]{|\det #1 |}

\newcommand{\galpha}{\grv{\alpha}}
\newcommand{\gbeta}{\grv{\beta}}
%\newcommand{\rnlessO}{\mb{R}^n \setminus \vec{0}}
\usepackage{listings}

\interfootnotelinepenalty=10000

\newcommand{\sectionline}{%
  \nointerlineskip \vspace{\baselineskip}%
  \hspace{\fill}\rule{0.5\linewidth}{.7pt}\hspace{\fill}%
  \par\nointerlineskip \vspace{\baselineskip}
}

\renewcommand{\labelenumii}{\roman{enumii})}

\begin{document}


\section*{Engineering Calculus (MA4003)}
\subsection*{Laplace Transforms (Qs 1 and 2)}
\begin{itemize}
\item[(a1)] Number each entry of the Laplace Transforms - you should have 22 entries. (e.g. Heaviside function is entry 16).
\item[(a2)] Questions 1 and 2 typically involve using one of the table entries to find the Laplace transform.
\end{itemize}
\subsection*{Inverse Laplace Transforms (Q3)}
\begin{itemize}
\item[(b1)] Consider function in form $f(t-a) \times u_a(t)$. The value of $a$ should be evident. Determine $f(t-a)$ and hence $f(t)$. From $f(t)$ compute $F(s)$.
\end{itemize}

\subsection*{Inverse Laplace Transforms (Qs 4 and 5)}
\begin{itemize}
\item[(c1)]
Always (attempt to) factorize the quadratic component.
\item[(c2)] Use table entries 8 and 9 for the following form: \[ \frac{s+k}{(s+a)(s+b)} = \left( \frac{s}{(s+a)(s+b)} \right) + k \times \left( \frac{1}{(s+a)(s+b)} \right)  \]
\item[(c3)] Factorize because sometimes terms cancel each other out.
\[ \frac{s+a}{s^2+(a+b)s + ab}  = \frac{s+a}{(s+a)(s+b)} = \frac{1}{s+b} \]
\item[(c4)]
Sometimes there is no obvious way to factorize the denominator. Try a different approach:

\[ \frac{s+a}{(s+a)^2 + m}  \mbox{ remark: We can use now shifting theorem : Table Entry 15} \]
\item[(c5)]
Numerator can be re-expressed as sum of two terms. $ s = (s-a) + a $
\[ \frac{s}{(s+a)^2} =  \frac{s-a}{(s+a)^2} + \frac{a}{(s+a)^2}  = \frac{1}{s+a} + \frac{a}{(s+a)^2} \]
\end{itemize}



\subsection*{Fourier Coefficients (Q9)}
\begin{itemize}
\item[(g1)]

\begin{tabular}{lcr}

  % after \\: \hline or \cline{col1-col2} \cline{col3-col4} ...
  $a_0 = \frac{1}{\pi}\int^{\pi }_{-\pi} f(x) dx$ & \mbox{     }$ a_n = \frac{1}{\pi}\int^{\pi }_{-\pi} f(x) cos(nx) dx $ & \mbox{     }$b_n = \frac{1}{\pi}\int^{\pi }_{-\pi} f(x) sin(nx) dx$ \\

\end{tabular}
\item[(g2)] Important - Revise ``Integration by Parts".
\end{itemize}
\newpage




%-----------------------------------------------------------------------------%
\section*{Question 2}
Choose appropriate Laplace Transform from tables. (Table Entries 3 and 4)
%----------------------------------------%
\subsection*{Q2 2011 Yellow}
$\mathcal{L}[e^{4t}(5t-1)]$ = $5\mathcal{L}[e^{4t}t]$ - $\mathcal{L}[e^{4t}]$
%----------------------------------------%
\subsection*{Q2 2011 Green}
$\mathcal{L}[e^{-5t}(2t+1)]$ = $2\mathcal{L}[e^{-5t}t]$ + $\mathcal{L}[e^{-5t}]$
%----------------------------------------%
\subsection*{Q2 2010 Yellow}

$\mathcal{L}[e^{3t}(t+1)]$ = $5\mathcal{L}[e^{3t}t]$ + $\mathcal{L}[e^{3t}]$
%----------------------------------------%
\subsection*{Q2 2010 Green}

$\mathcal{L}[e^{-2t}(t+2)]$ = $5\mathcal{L}[e^{-2t}t]$ + 2$\mathcal{L}[e^{-2t}]$

%----------------------------------------%
\newpage
%-----------------------------------------------------------------------------%
\section*{Question 3}
Important : Table Entry 17 (the one after the Heaviside Function)
%----------------------------------------%
\subsection*{2011 Yellow}
$g(t) = sin(2t - 6)u_3(t)$ is in form $f(t-a)u_a(t)$ , with $a=3$ \\
If $f(t-a) = f(t-3) = sin(2t - 6)$ then $f(t)= sin(2t)$ \\
$\mathcal{L}[f(t-a)u_a(t)]$ =$e^{-as}F(s)$ \\
$F(s)$ is simply $\mathcal{L}[f(t)]$\\
%----------------------------------------%
\subsection*{2011 Green}
$g(t) = cos(t-2)u_2(t)$ is in form $f(t-a)u_a(t)$ , with $a=2$ \\
If $f(t-a) = f(t-2) = cos(t-2)$ then $f(t)= cos(t)$ \\
$\mathcal{L}[f(t-a)u_a(t)]$ =$e^{-as}F(s)$ \\
$F(s)$ is simply $\mathcal{L}[f(t)]$\\
%----------------------------------------%
\subsection*{2010 Yellow}
$g(t) = cos(t-3)u_3(t)$ is in form $f(t-a)u_a(t)$ , with $a=3$ \\
If $f(t-a) = f(t-3) = cos(t-3)$ then $f(t)= cos(t)$ \\
$\mathcal{L}[f(t-a)u_a(t)]$ =$e^{-as}F(s)$ \\
$F(s)$ is simply $\mathcal{L}[f(t)]$\\
%----------------------------------------%
\subsection*{2010 Green}
$g(t) = sin(2t - 2)u_1(t)$ is in form $f(t-a)u_a(t)$ , with $a=1$ \\
If $f(t-a) = f(t-1) = sin(2t - 2)$ then $f(t)= sin(2t)$ \\
$\mathcal{L}[f(t-a)u_a(t)]$ =$e^{-as}F(s)$ \\
$F(s)$ is simply $\mathcal{L}[f(t)]$\\
%----------------------------------------%


%----------------------------------------%
\subsection*{Q5 2010 Yellow}
\begin{itemize}
\item Find the inverse Laplace transform of
\[ G(s) = \frac{s-2}{s^2 + s + 6} \]
\item Factorise the denominator:  $(s+2)(s+3)$
\item Can arrange expression in a way that Table Entries 8 and 9 can be used intuitively ($a=-3, b=-2$)
\[ G(s) = \frac{s-2}{(s+2)(s+3)} = \left(\frac{s}{(s+2)(s+3)} \right)-2\left(\frac{1}{(s+2)(s+3)}\right) \]
\end{itemize}
%----------------------------------------%
\subsection*{Q5 2010 Green}
\begin{itemize}
\item Find the inverse Laplace transform of
\[ G(s) = \frac{s-2}{s^2 - s - 6} \]
\item Factorise the denominator?  $(s+2)(s-3)$
\item Can arrange expression in a way that Table Entries 8 and 9 can be used intuitively ($a=3, b=-2$)
\[ G(s) = \frac{s-2}{(s+2)(s-3)} = \left(\frac{s}{(s+2)(s-3)} \right)-2\left(\frac{1}{(s+2)(s-3)}\right) \]
\end{itemize}
%----------------------------------------%
\section*{Question 6}

\subsection*{Q6 2011 Yellow}
\begin{itemize}
\item $f_1(t)=t^2$ : Laplace Transform $F_1(s)={2 \over s^3}$
\item $f_2(t)=t^2$ : Laplace Transform $F_2(s)={2 \over s^3}$
\item $\mathcal{L}[g(t)] = F_1(s)\times F_2(s) = {4 \over s^6}$
\item $g(t)$ = $\mathcal{L}^{-1}[{4 \over s^6}]$
\item Using Table Entry 3: $n$ is necessarily $5$ (5! = 120)
\[ \mathcal{L}[k.t^n] = k\frac{n!}{s^{n+1}} \]
\item $k.n! = 120$, $\therefore k=1/30$
\item Answer $f(t) = t^5/30$
\end{itemize}

%----------------------------------------%
\subsection*{Q6 2011 Green}
\begin{itemize}
\item $f_1(t)=e^{-t}$ : Laplace Transform $F_1(s)={1 \over s+1}$
\item $f_2(t)=e^{-t}$ : Laplace Transform $F_2(s)={1 \over s+1}$
\item $\mathcal{L}[g(t)] = F_1(s)\times F_2(s) = {1 \over (s+1)^2}$
\item $g(t)$ = $\mathcal{L}^{-1}[{1 \over (s+1)^2}]$
\item Using Table Entry 4: $n$ is necessarily $1$ , and $a=-1$
\item Answer $g(t) = te^{-t}$
\end{itemize}

%----------------------------------------%
\subsection*{Q6 2010 Yellow}

\begin{itemize}
\item $f_1(t)=t$ : Laplace Transform $F_1(s)={1 \over s^2}$
\item $f_2(t)=t^2$ : Laplace Transform $F_2(s)={2 \over s^3}$
\item $\mathcal{L}[g(t)] = F_1(s)\times F_2(s) = {2 \over s^5}$
\item $g(t)$ = $\mathcal{L}^{-1}[{2 \over s^5}]$
\item Using Table Entry 3: $n$ is necessarily $4$
\item Remember to use a scaling coefficient $k$ in conjunction with the formula.
\item $G(s) = k { 2 \over s^{4+1}}$
\end{itemize}
\subsection*{2010 Green}

\begin{itemize}
\item $f_1(t)=e^t$ : Laplace Transform $F_1(s)={1 \over s-1}$
\item $f_2(t)=e^{-t}$ : Laplace Transform $F_2(s)={2 \over s+1}$
\item $\mathcal{L}[g(t)] = F_1(s)\times F_2(s) = {1 \over s^2-1}$
\item $g(t)$ = $\mathcal{L}^{-1}[{1 \over s^2-1}]$
\item Using Table Entry 6: $g(t)$ = $sinh(t)$
\item $sinh(t)$ is not an option, but an equivalent expression is present.
\item \[ sinh(t) = \frac{e^t - e^{-t}}{2} \]
\end{itemize}
%----------------------------------------%
\newpage
%-------------------------------------------------------------------------%
\section*{Question 7}

\begin{itemize}
\item The period of a function $p$ may be identified by the following term : $f(t) = f(t+p)$
\item Period of a function is inversely proportional to the coefficient.
\item If $f(x)$ has period $p$ , then $f(2x)$ has period $p/2$
\item For older questions a periodic trigonometric function of form $trig(kx)$ has period $2\pi/k$
\end{itemize}
%----------------------------------------%
\subsection*{Q7 2011 Yellow}
\begin{itemize}
\item If $f(x)$ has period $2$ , then $f(2x)$ has period $1$
\end{itemize}
%----------------------------------------%
\subsection*{Q7 2011 Green}
\begin{itemize}
\item If $f(x)$ has period $2\pi$ , then $f(2x)$ has period $\pi$
\end{itemize}
%----------------------------------------%
\subsection*{Q7 2010 Yellow}
\begin{itemize}
\item  $trig(kx)$ $k$= ${pi\over 2}$
\item  $p$ = $\frac{2\pi}{pi / 2}$ =4
\end{itemize}
%----------------------------------------%
\subsection*{Q7 2010 Green}
\begin{itemize}
\item  $trig(kx)$ $k$= ${1\over2}$
\item  $p$ = $\frac{2\pi}{1/2}$ = $4\pi$
\end{itemize}
%----------------------------------------%
\newpage
\subsection{Q8 2010 Green}
\begin{itemize}
\item $f(x) = x - x^5$
\item $f(-1) = (-1) - (-1)^5$ = 0
\item $f(1) = (1) - (1)^5$ = 0
\item use a different number instead
\item $f(-1/2) = (-1/2) - (-1/2)^5$ = 0
\item $f(1/2) = (1/2) - (1/2)^5$ = 0
\item $f(x)$ is odd

\item $g(x) = x^2 sin x$
\item $g(-1) = (-1^2) \times sin(-1)$
\item $g(1) = (1^2) \times sin(-1)$
\item $g(x)$ is also odd
\end{itemize}
%----------------------------------------%
\newpage
\subsection*{Revision}
\Large{

\begin{eqnarray}
a_0 &=& \frac{1}{L}\int^{L }_{-L} f(x) dx \\
a_n &=& \frac{1}{L}\int^{L }_{-L} f(x) cos(nx) dx \\ 
b_n &=& \frac{1}{L}\int^{L }_{-L} f(x) sin(nx) dx
\end{eqnarray}

\begin{itemize}
\item \textbf{ ``a for even"}: If function is odd : $a_0 = 0$ and $a_n=0$
\item \textbf{ ``b for odd"}: If function is even : $b_n=0$
\end{itemize}
}
\subsection*{Q10 2011 Yellow} %Done
\Large{
\begin{itemize}
\item Re-express function \[f(x) = cosh(x) = \frac{e^{x} + e^{-x}}{2} \]
\item Range = -1 to 1. $L=1$ $\therefore 1/L = 1$ Also remark : this is an even function
\item \[\int\limits^{1}_{-1} f(x) dx = 2\int\limits^{1}_{0} f(x) dx = 2\int\limits^{1}_{0}\frac{e^{x}}{2} + \frac{e^{-x}}{2} dx \]
    \item \[ 2\int\limits^{1}_{0}\frac{e^{x}}{2} + \frac{e^{-x}}{2} dx  = 2\times \left[ \frac{e^{x}}{2} - \frac{e^{-x}}{2} \right]^{1}_{0} \]
\item remark : \[ \frac{e^{x}}{2} - \frac{e^{-x}}{2} = \frac{e^{x} - e^{-x}}{2}= sinh(x )\] 
\item Simplifying 
\[ 2\times \left[ \frac{e^{x}}{2} - \frac{e^{-x}}{2} \right]^{1}_{0}  = 2\times \left[sinh(1) - sinh(0)\right] = 2sinh(1) \]
\end{itemize}
}
\newpage
\subsection*{Q10 2011 Green} 
\Large{
\begin{itemize}
\item Re-express function \[f(x) = |x|  \]
\item Range = -1 to 1. $L=1$ ($\therefore 1/L = 1$) Also remark : this is an even function.
\item \[\int\limits^{1}_{-1} f(x) dx = \int\limits^{0}_{-1} (-x) dx + \int\limits^{1}_{0} (x) dx \]
\item \[\int\limits^{1}_{-1} f(x) dx = 2\int\limits^{1}_{0} f(x) dx = 2 \times \int\limits^{1}_{0} (x) dx \]

\item \[ 2 \times \int\limits^{1}_{0} x dx = 2 \times \left[ \frac{x^2}{2} \right]^{1}_{0} = 
2 \times \left[ \frac{1^2}{2} - \frac{0}{2} \right]  = 1 \] 
\end{itemize}
}

\subsection*{Q10 2010 Green}
\Large{
\begin{itemize}
\item Re-express function \[f(x) = xcos(x)  \]
\item This is an \textbf{ODD} function. $a_0$ is necessarily 0.
\item You can check this by trying out some trial values
\begin{itemize}
\item[*] $f(-\pi) = -\pi \times (-1) = \pi$
\item[*] $f(\pi) = \pi \times (-1) = -\pi$
\end{itemize}
\end{itemize}
}


\subsection*{Q10 2010 Green}
\Large{
\begin{itemize}
\item Re-express function \[f(x) = -x^3  \]
\item This is an \textbf{ODD} function. $a_0$ is necessarily 0.
\item Lets do it out just to make sure.
\item Range = -1 to 1. $L=1$ ($\therefore 1/L = 1$ )
\item \[\int\limits^{1}_{-1} f(x) dx =\int\limits^{1}_{-1}-x^3 dx = \left[\frac{-x^4}{4} \right]^{1}_{-1}  = \left[\left(- \frac{1}{4} \right) - \left(- \frac{1}{4} \right) \right]  = 0\]
\item Remark: Be VERY Careful with signs
\end{itemize}
}
%----------------------------------------%
\end{document} 
