
%=================================================================%
\newpage
 

Example 1

Find g(t), the inverse Laplace transform of G(s) =2s3

Consider this in form G(s) = -F(s)

Necessarily 

 

Finding the inverse Laplace transform of F(s)

 

f(t) = t

 


g(t) = tf(t) =t2

 

Summary:

·       Given G(s), we consider it in form G(s) =-F(s)

·       We negate G(s) and integrate it to find F(s).

·       We get the inverse laplace transform of F(s) to find f(t).

·       We multiply f(t) by ‘t’ to find g(t)

 
%=================================================================%
\newpage
Example 2

Find the inverse Laplace transform of lnss-1

 

To solve this we use a same approach to the one in the previous example, but in reverse.

 

·       We consider the given equation in form F(s).

·       We negate F(s) and differentiate it to find G(s).

·       We get the inverse Laplace transform of G(s) , yielding g(t).

·       We divide g(t) by ‘t’ to find f(t)

 

 F(s) = lnss-1= ln(s) - ln(s-1)           

G(s) =-F(s) = -dds( ln(s)) +dds(ln(s-1))


G(s) =dds( ln(s-1)) -dds(ln(s))


G(s) =1s-1-1s


g(t) =et- 1


f(t) =g(t)t=et- 1t

%=================================================================%
\newpage

 Example 3

G(s) = lns2s2-1


 lns2s2-1=ln(s2) -ln(s2-1)


ln(s2) = 2ln(s)



dds2ln(s) =2s



To compute  ddsln(s2- 1), apply the chain rule.



ddsln(s2- 1) = 2s1s2-1 


 F(s) =2ss2-1-2s



 g(t) = 2 cos(t) -2


f(t) =g(t)t=2 cos(t) -2t


%=================================================================%
\newpage
 

Example 4                  Find the inverse Laplace transform of  F(s)


F(s) =tan-1(s)


G(s) =-F(s) = -ddstan-1(s)


G(s) =-1s2-1


g(t) = -sin(t)


f(t) =g(t)t=-sin(t)t

 


