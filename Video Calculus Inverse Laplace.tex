
\documentclass{beamer}

\usepackage{amsmath}
\usepackage{amssymb}
\usepackage{graphics}
\usepackage{framed}

\begin{document}


%-----------------------------------------%
\begin{frame}
\Huge
\[\mbox{Calculus For Engineers}\]
\LARGE
\[\mbox{Inverse Laplace Transforms}\]

\Large
\[\mbox{kobriendublin.wordpress.com}\]
\[\mbox{Twitter: @StatsLabDublin}\]
\end{frame}

%-----------------------------------------%
\begin{frame}
\frametitle{Inverse Laplace Transforms }
\LARGE

Find the Inverse Laplace Transform of the following expression

\[ G(s) = \frac{7}{s^5} \]

\end{frame}

%-----------------------------------------%
\begin{frame}
\frametitle{Inverse Laplace Transforms}
\Large
The Laplace transform $G(s)$ is structured in the form:
\[ \frac{n!}{s^{(n+1)}} \]

Relevant entry from Laplace Transform formula sheet (entry no. 3):
\[ L[t^n] = \frac{n!}{s^{(n+1)}}  \]

\phantom{Clearly n = 4}
\bigskip
\end{frame}

%-----------------------------------------%
\begin{frame}
\frametitle{Inverse Laplace Transforms}
\LARGE
\vspace{-1.9cm}
\[ L[t^4] = \frac{4!}{s^5} =\frac{24}{s^5} \]

\[ k\times L[t^4] = k \times \frac{4!}{s^5} =\frac{24k}{s^5} \]

\bigskip
%Clearly n= 4 and 24k = 7

%Therefore $k = {7 \over 24}$
\end{frame}
\end{document}
%-----------------------------------------%
%-----------------------------------------%
% Example 2 [MT 2009 Q4] Find the inverse Laplace transform of G(s)
\begin{frame}
\frametitle{Inverse Laplace Transforms}
\[ G(s) = \frac{s}{s^2 -2s + 1} \]
Factorise the denominator: \[s ^2 - 2s + 1 = (s - 1)^2\]
\end{frame}
%-----------------------------------------%
\begin{frame} 
\frametitle{Inverse Laplace Transforms}
Rewrite G(s) as follows: \[G(s) = \frac{s}{(s-1)^2} \]

Can we use the form $F(s-a)$ with $a=1$
s = (s-1)+1
The expression is now in form $F(s-a)$, with $a = 1$.
\end{frame}
%-----------------------------------------%
\begin{frame} 
\frametitle{Inverse Laplace Transforms}
\[ F(s) = \frac{s+1}{s^2} = \frac{s}{s^2} + \frac{1}{s^2} = \frac{1}{s} + \frac{1}{s^2}\]

\[ f(t) = L^{-1}[\frac{1}{s}] + L^{-1}[\frac{1}{s^2}] = 1 + t\]

\[ L^{-1}[G(s)] = e^{-t}(t+1 ) \]

\end{frame}
%-----------------------------------------%

% [MT 2009 Q5] Find the inverse Laplace transform of G(s)
\begin{frame} 
\frametitle{Inverse Laplace Transforms}
\[ G s) = \frac{s+2}{s^2 + 5s -6} \]

\end{frame}
%-----------------------------------------%
\begin{frame}
\frametitle{Inverse Laplace Transforms}
Factorize the denominator: $s^2 + 5s -6$
\[= (s+1)(s-6) \]


\[ G s) = \frac{s+2}{s^2 + 5s -6} = \frac{s+2}{(s+1)(s-6)} \]
\end{frame}

%-----------------------------------------%
\begin{frame}
\frametitle{Inverse Laplace Transforms}
\[ G s)  = \frac{s+2}{(s+1)(s-6)}  = \frac{s}{(s+1)(s-6)} +  \frac{2}{(s+1)(s-6)}\]


let a = 1, b = - 6


\end{frame}
%-----------------------------------------%
\begin{frame}
\frametitle{Inverse Laplace Transforms}


\end{frame}

%-----------------------------------------%

% 
[MT 2008 Q5] Find the inverse Laplace transform of G(s)
\begin{frame}
\frametitle{Inverse Laplace Transforms}

\[G(s) = \frac{s-2}{s^2 -5s +6} \]


\end{frame}
%------------------------------------------%
\begin{frame}
Factorize the denominator: 

\[s^2 -5s +6 =(s-2)(s+3)\]

\[G(s) = \frac{s-2}{s^2 -5s +6} = \frac{s-2}{(s-2)(s+3)} \]

\end{frame}
%-------------------------------------------%
\begin{frame}
From entry no 4. the inverse Laplace transform is computed.

\end{frame}

\end{document}