

\documentclass[a4]{beamer}
\usepackage{amssymb}
\usepackage{graphicx}
\usepackage{subfigure}
\usepackage{newlfont}
\usepackage{amsmath,amsthm,amsfonts}
%\usepackage{beamerthemesplit}
\usepackage{pgf,pgfarrows,pgfnodes,pgfautomata,pgfheaps,pgfshade}
\usepackage{mathptmx}  % Font Family
\usepackage{helvet}   % Font Family
\usepackage{color}

\mode<presentation> {
 \usetheme{Default} % was Frankfurt
 \useinnertheme{rounded}
 \useoutertheme{infolines}
 \usefonttheme{serif}
 %\usecolortheme{wolverine}
% \usecolortheme{rose}
\usefonttheme{structurebold}
}

\setbeamercovered{dynamic}

\title[MathsCast]{MathsCasts - Dynamic Maths Support) \\ {\normalsize Engineering Maths 3 : Convolution}}
\author[Kevin O'Brien]{Kevin O'Brien \\ {\scriptsize Kevin.obrien@ul.ie}}
\date{Summer 2011}
\institute[Maths \& Stats]{Dept. of Mathematics \& Statistics, \\ University \textit{of} Limerick}

\renewcommand{\arraystretch}{1.5}

\begin{document}

%--------------------------------------------------------------------------------%
\frame{\frametitle{Convolution}
\Large
\begin{itemize}
\item Convolution is a mathematical operation on two functions $f(t)$ and $g(t)$, creating a third function that can be considered a ``blending" of the two component functions.


\item The convolution of functions is denoted $ (f * g )(t)$, and can be evaluated using this formula:
\[(f * g )(t) = \int_{-\infty}^{\infty} f(u)\, g(t - u)\, du\]
\item Convolution is quite useful in a lot of software and engineering applications, such as image processing.
\end{itemize}
}

%--------------------------------------------------------------------------------%
\frame{
\frametitle{Using Laplace Transforms}
\Large
We can compute $ (f * g )(t)$, the convolution of two functions $f(t)$ and $g(t)$, by following these steps:\\
\bigskip
%\normalsize
\begin{itemize}
\item Get the Laplace transforms of the two component functions : $\mathcal{L}[f(t)] = F(s)$ and $\mathcal{L}[g(t)] = G(s)$
\item Multiply these two Laplace transforms: $ F(s) \times G(s)$
\item Find the inverse Laplace transform of the product: $\mathcal{L}^{-1}[F(s) \times G(s)] $
%\item The inverse Laplace transform is the convolution, which is the result we are looking for.
\end{itemize}
}

%------------------------------------------------------------------------%
%Convolution
\frame{
\frametitle{Example 1}
\Large
Use Laplace transforms to compute $t \ast t^{2}$, the convolution of $t$ and $t^{2}$ \\ \bigskip

First compute the Laplace transforms of the two component functions:\\
\bigskip
% $\mathcal{L} [ t \ast t^{2} ]  = \mathcal{L} [ t ] \times \mathcal{L} [  t^{2} ]$\\
\begin{itemize}
\item $\mathcal{L} [ t ]$ \bigskip % = {1 \over s^2}$\\
\item $\mathcal{L} [ t^2 ]$ % = {2 \over s^3}$\\
\end{itemize}
% $\mathcal{L} [ t \ast t^{2} ] = {1 \over s^2} \times {2 \over s^3} = {2 \over s^5}$\\
}
\frame{
\frametitle{Example 1}
\Large
\vspace{-4cm}
The Laplace transform of the convolution is the product of the Laplace transforms of two component functions.

}
\frame{
\frametitle{Example 1}
\Large
\vspace{-4cm}
Compute the inverse Laplace transform to find the convolution of the functions.

}

\frame{
\frametitle{Example 1}
\Large
\vspace{-3.5cm}
Using the table of formulae:
\[ \mathcal{L}^{-1}\left[k \times { n! \over s^{n+1}}\right] = k \times t^{n}\]

}
%------------------------------------------------------------------------%
%Convolution
\frame{
\frametitle{Example 2}
\Large
Use Laplace transforms to compute $e^t \ast e^{-t}$, the convolution of $e^t$ and $e^{-t}$ \\ \bigskip
First compute the Laplace transforms of the two component functions:\\
\bigskip
% $\mathcal{L} [ t \ast t^{2} ]  = \mathcal{L} [ t ] \times \mathcal{L} [  t^{2} ]$\\
\begin{itemize}
\item $\mathcal{L} [ e^t ]$ \bigskip % = {1 \over s^2}$\\
\item $\mathcal{L} [ e^{-t} ]$ % = {2 \over s^3}$\\
\end{itemize}

}
\frame{
\frametitle{Example 2}
\Large
\vspace{-4cm}
The Laplace transform of the convolution is the product of the Laplace transforms of two component functions.

}
\frame{
\frametitle{Example 2}
\Large
\vspace{-4cm}
Compute the inverse Laplace transform to find the convolution of the functions.

}

\end{document} 