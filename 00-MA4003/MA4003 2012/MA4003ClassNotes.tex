\documentclass[12pt, a4paper]{article}
\usepackage{epsfig}
\usepackage{subfigure}
%\usepackage{amscd}
\usepackage{amssymb}
\usepackage{amsbsy}
\usepackage{amsthm, amsmath}
\usepackage[usenames]{color}
\usepackage{listings}
\lstset{% general command to set parameter(s)
basicstyle=\small, % print whole listing small
keywordstyle=\color{red}\itshape,
% underlined bold black keywords
commentstyle=\color{blue}, % white comments
stringstyle=\ttfamily, % typewriter type for strings
showstringspaces=false,
numbers=left, numberstyle=\tiny, stepnumber=1, numbersep=5pt, %
frame=shadowbox,
rulesepcolor=\color{black},
,columns=fullflexible
} %
%\usepackage[dvips]{graphicx}
\usepackage{natbib}
\bibliographystyle{chicago}
\usepackage{vmargin}
% left top textwidth textheight headheight
% headsep footheight footskip
\setmargins{1.25cm}{1.25cm}{18.5 cm}{25cm}{0.5cm}{0cm}{1cm}{1cm}
\renewcommand{\baselinestretch}{1.2}
\pagenumbering{arabic}
\theoremstyle{plain}
\newtheorem{theorem}{Theorem}[section]
\newtheorem{corollary}[theorem]{Corollary}
\newtheorem{ill}[theorem]{Example}
\newtheorem{lemma}[theorem]{Lemma}
\newtheorem{proposition}[theorem]{Proposition}
\newtheorem{conjecture}[theorem]{Conjecture}
\newtheorem{axiom}{Axiom}
\theoremstyle{definition}
\newtheorem{definition}{Definition}[section]
\newtheorem{notation}{Notation}
\theoremstyle{remark}
\newtheorem{remark}{Remark}[section]
\newtheorem{example}{Example}[section]
\renewcommand{\thenotation}{}
\renewcommand{\thetable}{\thesection.\arabic{table}}
\renewcommand{\thefigure}{\thesection.\arabic{figure}}

\begin{document}

\section*{Engineering Calculus (MA4003)}
\subsection*{Laplace Transforms (Qs 1 and 2)}
\begin{itemize}
\item[(a1)] Number each entry of the Laplace Transforms - you should have 22 entries. (e.g. Heaviside function is entry 16).
\item[(a2)] Questions 1 and 2 typically involve using one of the table entries to find the Laplace transform.
\end{itemize}
\subsection*{Inverse Laplace Transforms (Q3)}
\begin{itemize}
\item[(b1)] Consider function in form $f(t-a) \times u_a(t)$. The value of $a$ should be evident. Determine $f(t-a)$ and hence $f(t)$. From $f(t)$ compute $F(s)$.
\end{itemize}

\subsection*{Inverse Laplace Transforms (Qs 4 and 5)}
\begin{itemize}
\item[(c1)]
Always (attempt to) factorize the quadratic component.
\item[(c2)] Use table entries 8 and 9 for the following form: \[ \frac{s+k}{(s+a)(s+b)} = \frac{s}{(s+a)(s+b)} + \frac{k}{(s+a)(s+b)}  \]
\item[(c3)] Factorize because sometimes terms cancel each other out.
\[ \frac{s+a}{(s^2+(a+b)s + ab)}  = \frac{s+a}{(s+a)(s+b)} = \frac{1}{(s+b)} \]
\item[(c4)]
Sometimes there is no obvious way to factorize the denominator. Try a different approach:

\[ \frac{s+a}{(s+a)^2 + m}  \mbox{ remark: We can use now shifting theorem} \]
\item[(c5)]
Numerator can be re-expressed as sum of two terms. $ s = (s-a) + a $
\[ \frac{s}{(s+a)^2} =  \frac{s-a}{(s+a)^2} + \frac{a}{(s+a)^2}  = \frac{1}{s+a} + \frac{a}{(s+a)^2} \]
\end{itemize}

\subsection*{Convolution (Q6)}
\begin{itemize}
\item[(d1)] Find the Laplace transform of both terms individually : $F(s)$ and $G(s)$.
\item[(d2)] The Laplace transform of the convolution result $f(t) \ast g(t)$ is the product of $F(s)$ and $G(s)$.
\[ \mathcal{L} [ f(t) \ast g(t) ] = F(S)G(S) \]
\item[(d3)] To find $f(t) \ast g(t)$, Compute the inverse Laplace Transform of $F(S)\times G(S)$
\[ \mathcal{L}^{-1}[ F(S)G(S)] = f(t) \ast g(t) \]
\end{itemize}

\subsection*{Period of a function (old questions)}
\begin{enumerate}
\item Period of a function $Trig(kx)$:
\[ p =\frac{2\pi}{k} \]
\end{enumerate}
\subsection*{Even and Odd Functions (Q8)}
\begin{itemize}
\item[(f1)] Even Functions: $f(-a) = f(a)$
\item[(f2)] Odd Functions: $f(-a) = -f(a)$
\end{itemize}

\subsection*{Fourier Coefficients}
\begin{itemize}
\item[(g1)]

\begin{tabular}{lcr}
  
  % after \\: \hline or \cline{col1-col2} \cline{col3-col4} ...
  $a_0 = \frac{1}{\pi}\int^{\pi }_{-\pi} f(x) dx$ & \mbox{     }$ a_n = \frac{1}{\pi}\int^{\pi }_{-\pi} f(x) cos(nx) dx $ & \mbox{     }$b_n = \frac{1}{\pi}\int^{\pi }_{-\pi} f(x) sin(nx) dx$ \\

\end{tabular}
\item[(g2)] Important - Revise ``Integration by Parts".
\end{itemize}


\end{document}
