%\documentclass[11pt, a4paper,dalthesis]{report}    % final
%\documentclass[11pt,a4paper,dalthesis]{report}
%\documentclass[11pt,a4paper,dalthesis]{book}

\documentclass[11pt,a4paper,titlepage,oneside,openany]{article}

\pagestyle{plain}
%\renewcommand{\baselinestretch}{1.7}

\usepackage{setspace}
%\singlespacing
\onehalfspacing
%\doublespacing
%\setstretch{1.1}

\usepackage{amsmath}
\usepackage{amssymb}
\usepackage{amsthm}
\usepackage{multicol}

\usepackage[margin=3cm]{geometry}
\usepackage{graphicx,psfrag}%\usepackage{hyperref}
\usepackage[small]{caption}
\usepackage{subfig}

\usepackage{algorithm}
\usepackage{algorithmic}
\newcommand{\theHalgorithm}{\arabic{algorithm}}

\usepackage{varioref} %NB: FIGURE LABELS MUST ALWAYS COME DIRECTLY AFTER CAPTION!!!
%\newcommand{\vref}{\ref}

\usepackage{index}
\makeindex
\newindex{sym}{adx}{and}{Symbol Index}
%\newcommand{\symindex}{\index[sym]}
%\newcommand{\symindex}[1]{\index[sym]{#1}\hfill}
\newcommand{\symindex}[1]{\index[sym]{#1}}

%\usepackage[breaklinks,dvips]{hyperref}%Always put after varioref, or you'll get nested section headings
%Make sure this is after index package too!
%\hypersetup{colorlinks=false,breaklinks=true}
%\hypersetup{colorlinks=false,breaklinks=true,pdfborder={0 0 0.15}}


%\usepackage{breakurl}

\graphicspath{{./images/}}

\usepackage[subfigure]{tocloft}%For table of contents
\setlength{\cftfignumwidth}{3em}

\input{longdiv}
\usepackage{wrapfig}


%\usepackage{index}
%\makeindex
%\usepackage{makeidx}

%\usepackage{lscape}
\usepackage{pdflscape}
\usepackage{multicol}

\usepackage[utf8]{inputenc}

%\usepackage{fullpage}

%Compulsory packages for the PhD in UL:
%\usepackage{UL Thesis}
\usepackage{natbib}

%\numberwithin{equation}{section}
\numberwithin{equation}{section}
\numberwithin{algorithm}{section}
\numberwithin{figure}{section}
\numberwithin{table}{section}
%\newcommand{\vec}[1]{\ensuremath{\math{#1}}}

%\linespread{1.6} %for double line spacing

\usepackage{afterpage}%fingers crossed

\newtheorem{thm}{Theorem}[section]
\newtheorem{defin}{Definition}[section]
\newtheorem{cor}[thm]{Corollary}
\newtheorem{lem}[thm]{Lemma}

%\newcommand{\dbar}{{\mkern+3mu\mathchar'26\mkern-12mu d}}
\newcommand{\dbar}{{\mkern+3mu\mathchar'26\mkern-12mud}}

\newcommand{\bbSigma}{{\mkern+8mu\mathsf{\Sigma}\mkern-9mu{\Sigma}}}
\newcommand{\thrfor}{{\Rightarrow}}

\newcommand{\mb}{\mathbb}
\newcommand{\bx}{\vec{x}}
\newcommand{\bxi}{\boldsymbol{\xi}}
\newcommand{\bdeta}{\boldsymbol{\eta}}
\newcommand{\bldeta}{\boldsymbol{\eta}}
\newcommand{\bgamma}{\boldsymbol{\gamma}}
\newcommand{\bTheta}{\boldsymbol{\Theta}}
\newcommand{\balpha}{\boldsymbol{\alpha}}
\newcommand{\bmu}{\boldsymbol{\mu}}
\newcommand{\bnu}{\boldsymbol{\nu}}
\newcommand{\bsigma}{\boldsymbol{\sigma}}
\newcommand{\bdiff}{\boldsymbol{\partial}}

\newcommand{\tomega}{\widetilde{\omega}}
\newcommand{\tbdeta}{\widetilde{\bdeta}}
\newcommand{\tbxi}{\widetilde{\bxi}}



\newcommand{\wv}{\vec{w}}

\newcommand{\ie}{i.e. }
\newcommand{\eg}{e.g. }
\newcommand{\etc}{etc}

\newcommand{\viceversa}{vice versa}
\newcommand{\FT}{\mathcal{F}}
\newcommand{\IFT}{\mathcal{F}^{-1}}
%\renewcommand{\vec}[1]{\boldsymbol{#1}}
\renewcommand{\vec}[1]{\mathbf{#1}}
\newcommand{\anged}[1]{\langle #1 \rangle}
\newcommand{\grv}[1]{\grave{#1}}
\newcommand{\asinh}{\sinh^{-1}}

\newcommand{\sgn}{\text{sgn}}
\newcommand{\morm}[1]{|\det #1 |}

\newcommand{\galpha}{\grv{\alpha}}
\newcommand{\gbeta}{\grv{\beta}}
%\newcommand{\rnlessO}{\mb{R}^n \setminus \vec{0}}
\usepackage{listings}

\interfootnotelinepenalty=10000

\newcommand{\sectionline}{%
  \nointerlineskip \vspace{\baselineskip}%
  \hspace{\fill}\rule{0.5\linewidth}{.7pt}\hspace{\fill}%
  \par\nointerlineskip \vspace{\baselineskip}
}

\renewcommand{\labelenumii}{\roman{enumii})}

\begin{document}
\section*{Question 6}
Eigen Values, Eigen Vectors and Eigenspaces

\[
A = \left(
  \begin{array}{ccc}
 0&  1&  0\\
 0&  0&  1\\
 0&  2& -1\\
  \end{array}
\right)
\]
 
The eigenvalues are the solution to the following equation \[ \mbox{DET}(A-I\lambda) = 0 \].
Equivalently: $ \mbox{DET}(I\lambda-A) = 0 $
\[
\left|
  \begin{array}{ccc}
    0-\lambda & 1 & 0  \\
    0 & 0-\lambda & 1  \\
    0 & 2 & -1-\lambda  \\
  \end{array}
\right|
\]
Computing the determinant
\begin{itemize}
\item Pick a row or column (preferably one with lots of zeroes)
\item For each element in the row (or column) compute the 
\item
\end{itemize}


Show that A is diagonalizable over the reals and find the matrix P which diagonalises it. Hence or otherwise solve the system of differential equations
\newpage




\section*{Question 6}
Eigen Values, Eigen Vectors and Eigenspaces
\[
\left|
  \begin{array}{ccc}
    0-\lambda & 1 & 3  \\
    1 & 3-\lambda & 1  \\
    2 & 4 & 1-\lambda  \\
  \end{array}
\right|
\]
Show that A is diagonalizable over the reals and find the matrix P which diagonalises it. Hence or otherwise solve the system of differential equations
\newpage

\section*{Question 7}
\large{
\begin{itemize}
\item Solving a system of Linear Equations
\item LU decomposition of a matrix
\end{itemize}
%----------------------------------------%
LU decomposition
\[
\left(
  \begin{array}{ccc}
 1.000& -3.000&  1.000\\
 2.000& -5.000&  4.000\\
 2.000& -2.000& 11.000\\
  \end{array}
\right)
\]
%----------------------------------------%
Question 7 L matrix
\[
\left(
  \begin{array}{ccc}
 1.000&  0.000&  0.000\\
 1.000&  1.000&  0.000\\
 0.500& -0.167&  1.000\\
  \end{array}
\right)
\]

%----------------------------------------%
Question 7 U matrix
\[
\left(
  \begin{array}{ccc}
 2.000 &-5.000 & 4.000\\
 0.000 & 3.000 &7.000\\
 0.000 & 0.000 &0.167\\
  \end{array}
\right)
\]

\newpage
Define the condition number of a square matrix. Of what is it a measure ?
For the matrix A of part (a), use the LU decomposition or otherwise
to calculate its inverse, and then calculate its condition number using
the maximum row-sum norm.


%----------------------------------------%

} %end of large font
\end{document} 