%\documentclass[11pt, a4paper,dalthesis]{report}    % final
%\documentclass[11pt,a4paper,dalthesis]{report}
%\documentclass[11pt,a4paper,dalthesis]{book}

\documentclass[11pt,a4paper,titlepage,oneside,openany]{article}

\pagestyle{plain}
%\renewcommand{\baselinestretch}{1.7}

\usepackage{setspace}
%\singlespacing
\onehalfspacing
%\doublespacing
%\setstretch{1.1}

\usepackage{amsmath}
\usepackage{amssymb}
\usepackage{amsthm}
\usepackage{multicol}

\usepackage[margin=3cm]{geometry}
\usepackage{graphicx,psfrag}%\usepackage{hyperref}
\usepackage[small]{caption}
\usepackage{subfig}

\usepackage{algorithm}
\usepackage{algorithmic}
\newcommand{\theHalgorithm}{\arabic{algorithm}}

\usepackage{varioref} %NB: FIGURE LABELS MUST ALWAYS COME DIRECTLY AFTER CAPTION!!!
%\newcommand{\vref}{\ref}

\usepackage{index}
\makeindex
\newindex{sym}{adx}{and}{Symbol Index}
%\newcommand{\symindex}{\index[sym]}
%\newcommand{\symindex}[1]{\index[sym]{#1}\hfill}
\newcommand{\symindex}[1]{\index[sym]{#1}}

%\usepackage[breaklinks,dvips]{hyperref}%Always put after varioref, or you'll get nested section headings
%Make sure this is after index package too!
%\hypersetup{colorlinks=false,breaklinks=true}
%\hypersetup{colorlinks=false,breaklinks=true,pdfborder={0 0 0.15}}


%\usepackage{breakurl}

\graphicspath{{./images/}}

\usepackage[subfigure]{tocloft}%For table of contents
\setlength{\cftfignumwidth}{3em}

\input{longdiv}
\usepackage{wrapfig}


%\usepackage{index}
%\makeindex
%\usepackage{makeidx}

%\usepackage{lscape}
\usepackage{pdflscape}
\usepackage{multicol}

\usepackage[utf8]{inputenc}

%\usepackage{fullpage}

%Compulsory packages for the PhD in UL:
%\usepackage{UL Thesis}
\usepackage{natbib}

%\numberwithin{equation}{section}
\numberwithin{equation}{section}
\numberwithin{algorithm}{section}
\numberwithin{figure}{section}
\numberwithin{table}{section}
%\newcommand{\vec}[1]{\ensuremath{\math{#1}}}

%\linespread{1.6} %for double line spacing

\usepackage{afterpage}%fingers crossed

\newtheorem{thm}{Theorem}[section]
\newtheorem{defin}{Definition}[section]
\newtheorem{cor}[thm]{Corollary}
\newtheorem{lem}[thm]{Lemma}

%\newcommand{\dbar}{{\mkern+3mu\mathchar'26\mkern-12mu d}}
\newcommand{\dbar}{{\mkern+3mu\mathchar'26\mkern-12mud}}

\newcommand{\bbSigma}{{\mkern+8mu\mathsf{\Sigma}\mkern-9mu{\Sigma}}}
\newcommand{\thrfor}{{\Rightarrow}}

\newcommand{\mb}{\mathbb}
\newcommand{\bx}{\vec{x}}
\newcommand{\bxi}{\boldsymbol{\xi}}
\newcommand{\bdeta}{\boldsymbol{\eta}}
\newcommand{\bldeta}{\boldsymbol{\eta}}
\newcommand{\bgamma}{\boldsymbol{\gamma}}
\newcommand{\bTheta}{\boldsymbol{\Theta}}
\newcommand{\balpha}{\boldsymbol{\alpha}}
\newcommand{\bmu}{\boldsymbol{\mu}}
\newcommand{\bnu}{\boldsymbol{\nu}}
\newcommand{\bsigma}{\boldsymbol{\sigma}}
\newcommand{\bdiff}{\boldsymbol{\partial}}

\newcommand{\tomega}{\widetilde{\omega}}
\newcommand{\tbdeta}{\widetilde{\bdeta}}
\newcommand{\tbxi}{\widetilde{\bxi}}



\newcommand{\wv}{\vec{w}}

\newcommand{\ie}{i.e. }
\newcommand{\eg}{e.g. }
\newcommand{\etc}{etc}

\newcommand{\viceversa}{vice versa}
\newcommand{\FT}{\mathcal{F}}
\newcommand{\IFT}{\mathcal{F}^{-1}}
%\renewcommand{\vec}[1]{\boldsymbol{#1}}
\renewcommand{\vec}[1]{\mathbf{#1}}
\newcommand{\anged}[1]{\langle #1 \rangle}
\newcommand{\grv}[1]{\grave{#1}}
\newcommand{\asinh}{\sinh^{-1}}

\newcommand{\sgn}{\text{sgn}}
\newcommand{\morm}[1]{|\det #1 |}

\newcommand{\galpha}{\grv{\alpha}}
\newcommand{\gbeta}{\grv{\beta}}
%\newcommand{\rnlessO}{\mb{R}^n \setminus \vec{0}}
\usepackage{listings}

\interfootnotelinepenalty=10000

\newcommand{\sectionline}{%
  \nointerlineskip \vspace{\baselineskip}%
  \hspace{\fill}\rule{0.5\linewidth}{.7pt}\hspace{\fill}%
  \par\nointerlineskip \vspace{\baselineskip}
}

\renewcommand{\labelenumii}{\roman{enumii})}

\begin{document}




%----------------------------------------%
\section*{Question 3}
{\Large
\begin{itemize}
\item Fourier Series
\item Differential Equation
\end{itemize}
%----------------------------------------%
\begin{eqnarray}
a_0 &=& \frac{1}{L}\int^{L }_{-L} f(x) dx \\
a_n &=& \frac{1}{L}\int^{L }_{-L} f(x) cos(\frac{n\pi x}{L}) dx \\
b_n &=& \frac{1}{L}\int^{L }_{-L} f(x) sin(\frac{n\pi x}{L}) dx
\end{eqnarray}

\begin{itemize}
\item \textbf{ ``a for even"}: If function is odd : $a_0 = 0$ and $a_n=0$
\item \textbf{ ``b for odd"}: If function is even : $b_n=0$
\end{itemize}
}
%----------------------------------------%
\newpage
{\Large
\[ I_2 =  \int^{\pi}_{-\pi} f(x)cos(nx) dx \]
This integral can be simplifies as follows ( on acccount of the fact that it is even)

\[ I_2 =  2 \times \int^{\pi}_{0} f(x)cos(nx) dx \]

\[ I_2 =  2 \times \left( \int^{\pi/2}_{0} 0 \times cos(nx) dx + \int^{\pi}_{pi/2} 1 \times cos(nx) dx \right) \]

\[ I_2 =  2 \times  \int^{\pi}_{\pi/2}  cos(nx) dx  \]

\[ I_2 =  2 \left[  \frac{-sin(nx)}{n} \right]^{\pi}_{\pi/2}   \]

\[ I_2 =  2 \left[  \frac{-sin(n\pi)}{n} - \frac{-sin(n\pi/2)}{n} \right]   \]

$sin(n\pi) = 0$


\[ I_2 =  \frac{2sin(n\pi/2)}{n}   \]

\[ I_1 =  \int^{\pi}_{-\pi} f(x) dx \]

\[ I_2 =  2 \times \left( \int^{\pi/2}_{0} 0 \times cos(nx) dx + \int^{\pi}_{pi/2} 1 \times cos(nx) dx \right) \]

\[I_2 =\frac{2sin(n\pi /2)}{n \pi}   \]

\[a_n = \frac{I_2}{L} =\frac{2sin(n\pi /2)}{n \pi}   \]
}
\newpage
{\Large
\textbf{Computing $a_0$}

\[ I_1 =  \int^{\pi}_{-\pi} f(x) dx \]

\[ I_1 =  2 \times \left(  + \int^{\pi}_{pi/2} 1 \times  dx \right) \]

\[ I_1 =  2 \times \left(x \right)^{\pi}_{\pi/2} = \pi  \]

\[a_0 = \frac{I_2}{L} = 1   \]
{
\newpage
\subsection*{Evaluation of a series}
Hence evaluate the series \[1 - \frac{1}{3} + \frac{1}{5} + \frac{1}{7} + \ldots \]
\subsection*{Differential equation}
Use the Fourier series of part (a) to find a particular solution of the
differential equation
%----------------------------------------%
\end{document} 